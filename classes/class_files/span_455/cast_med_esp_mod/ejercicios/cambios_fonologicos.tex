\documentclass[12pt]{article}
\usepackage{tipa}
\usepackage[english,spanish]{babel}
\usepackage[utf8]{inputenc}
\usepackage{t1enc}

\usepackage[hmargin=2cm,vmargin=2cm]{geometry}
\geometry{a4paper}

\usepackage{fancyhdr} % This should be set AFTER setting up the page geometry
\pagestyle{fancy} % options: empty , plain , fancy
\renewcommand{\headrulewidth}{0pt} % customise the layout...
\lhead{}\chead{}\rhead{\large{Cambios fonológicos}}
\lfoot{}\cfoot{}\rfoot{}


\begin{document}


{\large{
\begin{enumerate}
\itemsep=.5mm
	\item Pérdida de \textipa{[m]} final
	\item Pérdida de \textipa{[h]} inicial
	\item Confluencias vocálicas
	\item Desvelaricación de \textipa{[w]} (> \ \textipa{[B]})
	\item {\textipa{[e]}} en hiato > \ \textipa{[j]}
	\item {\textipa{[t]}} y \textipa{[k]} ante \textipa{[j]} (> \ \textipa{[\textteshlig]} > \ \textipa{[ts]})
	\item {\textipa{[k]}} ante vocales anteriores (> \ \textipa{[\textteshlig]} > \ \textipa{[ts]})
	\item Pérdida de vocales intertónicas (primera fase)
	\item Palatalización de consonantes velares agrupadas
	\item Asimilación de ciertos grupos consonánticos
	\item Palatalización de consonantes ante \textipa{[j]}
	\item Diptongación
	\item Cambio de \textipa{[f]} inicial a \textipa{[h]}
	\item Rehilamiento de \textipa{[\textturny] en [\textyogh]}
	\item Lenición
	\item Palatalización de \textipa{[l]} y \textipa{[n]} geminadas
	\item Palatalización de \textipa{[kl], [pl], [fl]} en posición inicial
	\item Pérdida de vocales intertónicas (segunda fase)
	\item Pérdida de \textipa{[t], [d], [k]} finales
	\item Pérdida de \textipa{[e]} final
	\item Ajuste de grupos consonáticos
	\item Prótesis ante \textipa{[s]} agrupada incial
\end{enumerate}

\hrule

\begin{enumerate}
	\item[23.] Pérdida de \textipa{[h]} inicial, proveniente de \textipa{[f]}
	\item[24.] \textipa{/b/} y \textipa{/\textbeta/} confluyen en \textipa{/b/}  
	\item[25.] Desafricación de \textipa{[ts]} y \textipa{[dz]}  
	\item[26.] Ensordecimiento de las sibilantes sonoras  
	\item[27.] Cambio de articulación de \textipa{[\c{s}]}  
	\item[28.] Cambio de articulación de \textipa{[\textesh]} Se hace velar o incluso uvular  
	\item[29.] Yeísmo (\textipa{/\textturny/} > \textipa{/\textctj/})
\end{enumerate}
}}


\clearpage


\noindent \textbf{Aplica, en orden cronológico los 29 cambios fonológicos presentados por Pharies a las palabras siguientes de modo que ilustren la evolución fonológica de las mismas entre el latín común y el español moderno:} \\


\begin{tabular}{lllcl}
	    &                  & {\sc Forma} & {\sc N$^o$ del cambio} & {\sc Explicación} \\ [1ex]
	1.  & aliu `ajo'       & \textipa{[\textprimstress a.li.u]} & & Forma latina\\
	    &                  & \textipa{[\textprimstress a.le.o]} & 3 & Confluencias vocálicas \\ 
	    &                  & \textipa{[\textprimstress a.ljo]}  & 5 & [e] en hiajo > \ [j] \\ 
	    &                  & \textipa{[\textprimstress a.\textturny o]} & 11 & Palatalización de con. ante [j] \\ 
	    &                  & \textipa{[\textprimstress a.\textyogh o]} & 14 & Rehilamiento de [\textturny] en [\textyogh] \\
	    & 				   & \textipa{[\textprimstress a.\textesh o]} & 26 & Ensordecimiento de sibilantes sonoras \\
	    &                  & \textipa{[\textprimstress a.xo]} & 28 & Cambio de articulación de \textipa{[\textesh]} \\  [2ex]

	2.  & anniculu `añejo' & \textipa{[an.\textprimstress ni.ku.lu ]} & & Forma latina \\ [12ex]

	3.  & apicula `abeja' & \textipa{[a.\textprimstress pi.ku.la]} & & Forma latina\\ [12ex]

	4.  & auricula `oreja' & \textipa{[aw.\textprimstress \textfishhookr i.ku.la]} & & Forma latina \\ [12ex]

	5. & clausa `llosa'   & \textipa{[\textprimstress klaw.sa]} & & Forma latina \\ [12ex]

	6. & cond\={u}x\={\i} `conduje' & \textipa{[kon.\textprimstress du:k.si:]} & & Forma latina \\ [12ex]

	7. & corticia `corteza'  & \textipa{[ko\textfishhookr.\textprimstress ti.ki.a]} & & Forma latina \\ [12ex]

	8. & decimu `diezmo'     & \textipa{[\textprimstress de.ki.mu]} & & Forma latina \\



\end{tabular}

\clearpage

\begin{tabular}{lllcl}
	    &                  & {\sc Forma} & {\sc N$^o$ del cambio} & {\sc Explicación} \\ [1ex]
	9.	& d\={\i}cit `dice' & \textipa{[\textprimstress di:.kit]} & & Forma latina \\ [15ex]

	10.	& facit `hace' & \textipa{[\textprimstress fa.kit]} & & Forma latina \\ [15ex]

	11.	& lumbr\={\i}ce `lombriz' & \textipa{[lum.\textprimstress b\textfishhookr i.ke]} & & Forma latina \\ [15ex]

	12.	& malitia `maleza' & \textipa{[ma.\textprimstress li.ti.a]} & & Forma latina \\ [15ex]

	13.	& martiu `marzo' & \textipa{[\textprimstress ma\textfishhookr.ti.u]} & & Forma latina \\ [15ex]

	14.	& meliore `mejor' & \textipa{[me.\textprimstress li.o.\textfishhookr e]} & & Forma latina \\ [15ex]

	15.	& paus\={a}re `posar' & \textipa{[paw.\textprimstress sa:.\textfishhookr e]} & & Forma latina \\

\end{tabular}



\end{document}